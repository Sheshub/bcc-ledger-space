\section{Rewards and the Epoch Boundary}
\label{sec:epoch}

\newcommand{\UTxOEpState}{\type{UTxOEpState}}
\newcommand{\Acnt}{\type{Acnt}}
\newcommand{\PlReapState}{\type{PlReapState}}
\newcommand{\NewPParamEnv}{\type{NewPParamEnv}}
\newcommand{\Snapshots}{\type{Snapshots}}
\newcommand{\SnapshotEnv}{\type{SnapshotEnv}}
\newcommand{\SnapshotState}{\type{SnapshotState}}
\newcommand{\NewPParamState}{\type{NewPParamState}}
\newcommand{\EpochState}{\type{EpochState}}
\newcommand{\BlocksMade}{\type{BlocksMade}}
\newcommand{\Snapshot}{\type{Snapshot}}
\newcommand{\RewardUpdate}{\type{RewardUpdate}}

\newcommand{\obligation}[4]{\fun{obligation}~ \var{#1}~ \var{#2}~ \var{#3}~ \var{#4}}
\newcommand{\reward}[7]{\fun{reward}
  ~ \var{#1}~ \var{#2}~ \var{#3}~ \var{#4}~ \var{#5}~ \var{#6}~ \var{#7}}
\newcommand{\rewardOnePool}[9]{\fun{rewardOnePool}
  ~\var{#1}~\var{#2}~\var{#3}~\var{#4}~\var{#5}~\var{#6}~\var{#7}~\var{#8}~\var{#9}}
\newcommand{\isActive}[4]{\fun{isActive}~ \var{#1}~ \var{#2}~ \var{#3}~ \var{#4}}
\newcommand{\activeStake}[5]{\fun{activeStake}~ \var{#1}~ \var{#2}~ \var{#3}~ \var{#4}~ \var{#5}}
\newcommand{\poolRefunds}[3]{\fun{poolRefunds}~ \var{#1}~ \var{#2}~ \var{#3}}
\newcommand{\poolStake}[3]{\fun{poolStake}~ \var{#1}~ \var{#2}~ \var{#3}}
\newcommand{\stakeDistr}[3]{\fun{stakeDistr}~ \var{#1}~ \var{#2}~ \var{#3}}
\newcommand{\lReward}[4]{\fun{r_{operator}}~ \var{#1}~ \var{#2}~ \var{#3}~ {#4}}
\newcommand{\mReward}[4]{\fun{r_{member}}~ \var{#1}~ \var{#2}~ \var{#3}~ {#4}}
\newcommand{\poolReward}[5]{\fun{poolReward}~\var{#1}~{#2}~\var{#3}~\var{#4}~\var{#5}}
\newcommand{\createRUpd}[2]{\fun{createRUpd}~\var{#1}~\var{#2}}

In order to handle rewards and staking, we must change the stake distribution
calculation function to add up only the Bcc in the UTxO
 before performing any calculations. In Figure~\ref{fig:functions:stake-distribution}
 below, we do so using the function $\fun{utxoBcc}$, which returns the amount of Bcc
 tokens in an address.

%%
%% Figure Functions for Stake Distribution
%%
\begin{figure}[htb]
  \emph{Helper function}
  %
  \begin{align*}
    & \fun{utxoBcc} \in \UTxO \to \Addr \to \Coin \\
    & \fun{utxoBcc}~{\var{utxo}}~\var{addr} ~=~\sum_{\var{out} \in \range \var{utxo}, \fun{getAddr}~\var{out} = \var{addr}} \fun{getCoin}~\var{out}
  \end{align*}
  %
  \emph{Stake Distribution (using functions and maps as relations)}
  %
  \begin{align*}
      & \fun{stakeDistr} \in \UTxO \to \DState \to \PState \to \Snapshot \\
      & \fun{stakeDistr}~{utxo}~{dstate}~{pstate} = \\
      & ~~~~ \big((\dom{\var{activeDelegs}})
      \restrictdom\left(\fun{aggregate_{+}}~\var{stakeRelation}\right),
    ~\var{delegations},~\var{poolParams}\big)\\
      & \where \\
      & ~~~~ (~\var{rewards},~\var{delegations},~\var{ptrs},~\wcard,~\wcard,~\wcard)
        = \var{dstate} \\
      & ~~~~ (~\var{poolParams},~\wcard,~\wcard) = \var{pstate} \\
      & ~~~~ \var{stakeRelation} = \left(
        \left(\fun{stakeCred_b}^{-1}\cup\left(\fun{addrPtr}\circ\var{ptr}\right)^{-1}\right)
        \circ\left(\hldiff{\fun{utxoBcc}}~{\var{utxo}}\right)
        \right)
        \cup \var{rewards} \\
      & ~~~~ \var{activeDelegs} =
               (\dom{rewards}) \restrictdom \var{delegations} \restrictrange (\dom{poolParams})
  \end{align*}

  \caption{Stake Distribution Function}
  \label{fig:functions:stake-distribution}
\end{figure}
